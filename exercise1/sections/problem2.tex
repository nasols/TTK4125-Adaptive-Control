\section{Problem 2}

For problem 2 we are given a function $V(t)\geq 0 \forall t \geq 0 $, meaning its always positive or zero. 
In addition the problem states that the function is non-increasing, meaning that $\dot{V}(t) \leq 0 \forall t \geq 0$. 
This means that the function starts at some positive value $V(0) = c \geq 0$ and then either stays constant or decreases, either way this becomes an upper bound, $V_0$.

Since we are working with a non-increasing, bounded, function, and with the Bolzano-Weierstrass theorem, we know that a subsequence of $V(t)$ is converging to something.
Combined with the fact that $V(t)$ is non-increasing, we can say that when the subsequence converges, the rest of the sequence cannot rise from this point. 
Therefore the function must converge to some limit. I.e $\lim_{t\rightarrow \infty V(t)}$ exists.