\section{Problem 1}

\begin{itemize}
    \item a) 
    In the early days of adaptive control, one of the main problems they wanted to solve was flight controllers. 
    The dynamics of an aricraft are nonlinear and modelling the aicraft is very difficult, and their dynamics change
    depending on the operating point of the craft. The traditional control methods with static gains could not attack this problem.

    \item b)    
    The main difference between adaptive and robust control is that robust control tries to create a controller working within ranges that the system is expected to work within. 
    In these applications, the system parameters are know to a degree in which adaptive, or online, control is not needed. 
    In adaptive control, the system parameters ar unknown, and the goal of the controller is to change and adapt to these. 

    \item c) 
    The main disadvantages of gain scheduling is that the switch between different gains introduce a discontinuity into the system, which again can cause instabilities. 
    This is especially true if the gains are very different, or if the system changes rapidly between operating points, making the scheduling happen at a high frequency.
    This is often addressed by either interpolating between the gains or by introducing several operating points to make for a smoother transition. 
    Another disadvantage is that all these gain sets are pre-calculated, so the controller works offline. If the system encounters 
    big changes, in its dynamics or its environment, this could not be accounted for in the pre-computed gains. In this case the system can't 
    adapt to these changes and the performance will degrade. 

    \item d) 
    In direct adaptive control the controller parameters are adjusted directly to meet performance metrics. Therefore, skipping the model estimaton 
    and rather adjusting the controller to follow some refrence model directly.
    In indirect adaptive control, the system first estimates the system parameters then estimating the controller parameters based on these to meet requirements. 
    DIAGRAM HERE 

    \item e)
    For a LTI system to be minimum phase, it has all zeros with real part in the left half plane. 

    \item f)
    The certainty principle is that we assume the estimated parameters to be the true parameters of the system, and calculate controller parameters thereafter. 
    We dont make any further assumptions on the uncertainty of the parameters. 

    \item g) 
    Some basic on-line parameter estimation methods are: 
    \begin{itemize}
        \item Kalman filters
        \item Recursive least squares
        \item Gradient methods
        \item Lyapunov-based update laws
    \end{itemize}







\end{itemize}